
\documentclass[a4paper,11pt, oneside]{article}
\input{/Users/bryan/style.tex}

\title{Collision Thoughts}
%\author{Bryan Kaperick}

\newcommand{\mtot}{m_A + m_B}


\begin{document}
\maketitle

We have two objects, $A$ and $B$ which collide.  We assume the two objects are uniform density, rigid bodies, and convex polygons.

\section{Two balls collide in 1D space}
\begin{align*}
    p_A^{0} &= m_Av_A^0 & \text{(Momentum of object A)}\\
    p_B^{0} &= m_Bv_B^0 & \text{(Momentum of object B)}\\
    c_r &= \frac{v_B^1 - v_B^0}{v_A^1 - v_A^0}  &\text{(Coefficient of restitution)}.
\end{align*}
The coefficient of restitution, $c_r\in[0,1]$, gives a measure of elasticity of the collision. $c_r = 0$ denotes a completely inelastic collision (no change in velocity occurs) and $c_r = 1$ denotes a completely elastic collision (kinetic energy is conserved).

Given \emph{a priori} values for $m_A, m_B, v_A^0, v_B^0, c_r$, we are interested in determining $v_A^1$ and $v_B^1$.  To do this, we solve the linear system
\begin{align*}
    m_Av_A^0 + m_Bv_B^0 &= m_Av_A^1 + m_Bv_B^1\\
    c_r &= \frac{v_B^1 - v_B^0}{v_A^1 - v_A^0}.
\end{align*}

This yields
\begin{align*}
    v_A^1 &= \frac{m_Av_A^0 + m_Bv_B^0 + c_r m_B (v_B^0 - v_A^0)}{m_A + m_B}\\
    v_B^1 &= \frac{m_Bv_B^0 + m_Av_A^0 + c_r m_A (v_A^0 - v_B^0)}{m_B + m_A}.
\end{align*}

Intuitively, we hope the outcome does not explicitly depend on the velocities of the objects, instead in terms of momentum and mass only.  We can refomulate the results as follows.  (With $\mtot = m_A + m_B$.)
\begin{align*}
    p_A^1 &= \left((1-c_r)p_A^0 + (1+c_r)p_B^0\right)\left(\frac{m_A}{\mtot}\right)\\
    p_B^1 &= \left((1+c_r)p_A^0 + (1-c_r)p_B^0\right)\left(\frac{m_B}{\mtot}\right).
\end{align*}

\subsection{Sanity Checks on some example cases:}
\begin{enumerate}
    \item $m_A = m_B =:m,\quad v_A^0 = -v_B^0 =: v^0$
\begin{align*}
    p_A^1 &= -c_r p^0\\
    p_B^1 &= c_r p^0.
\end{align*}
    
    Here, the interpretation of $c_r$ is upheld as the linear damping of the final momenta of two identical objects with equi-opposite initial momenta.
    
    \item $m_A = m_B =:m$
\begin{align*}
    p_A^1 &= \frac{1-c_r}{2}p_A^0 + \frac{1+c_r}{2}p_B^0\\
    p_B^1 &= \frac{1+c_r}{2}p_A^0 + \frac{1-c_r}{2}p_B^0\\
\end{align*}

The resulting momenta are weighted averages of the initial momenta.  When $c_r = 1$, the two momenta swap so that $p_B^1 = p_A^0$ and vice versa.  When $c_r = 0$, the resulting momenta are equal since the two objects are same mass and are stuck together.

    \item $p_A^0 = -p_B^0 =: p^0$
\begin{align*}
    p_A^1 &= -2c_rp^0\left(\frac{m_A}{m_A + m_B}\right)\\
    p_B^1 &= 2c_rp^0\left(\frac{m_B}{m_A + m_B}\right)\\
\end{align*}
Both objects entered the collision with equi-opposite momentum entering the collision, and so both objects leave the collision with equi-opposite momentum, whose magnitude is linearly damped by the coefficient of restitution, and scaled by what proportion of the total mass that object accounted for.

\item $c_r = 1$ (Perfectly elastic collision)
\begin{align*}
    p_A^1 &= 2p_B^0\left(\frac{m_A}{\mtot}\right)\\
    p_B^1 &= 2p_A^0\left(\frac{m_B}{\mtot}\right).
\end{align*}

Recall since this is a 1D collision, the two objects' momenta had opposite directions initially.  This result shows in a perfectly elastic collision, the objects reverse directions and the magnitude of the resulting momentum is linearly proportional to the proportion of the total mass that object held.

\item $c_r = 0$ (Perfectly inelastic collision)
\begin{align*}
    p_A^1 &= (p_A^0 + p_B^0)\left(\frac{m_A}{\mtot}\right)\\
    p_B^1 &= (p_A^0 + p_B^0)\left(\frac{m_B}{\mtot}\right).
\end{align*}

Here, the two objects have the same velocity since they are stuck together, so their final momenta vary only by what their difference in mass is.



\end{enumerate}


\end{document}
